\subsection{Dspace}\label{dspace}

\begin{wrapfigure}{r}{0.2\textwidth} %this figure will be at the right
    \centering
    \includegraphics[width=0.2\textwidth]{../images/dspace.png}
\end{wrapfigure}
DSpace é um software para gerenciamento de acervo digital.

Como é um sistema em Java, ele está sendo executado na pasta \textit{webapp} do Tomcat, podendo ser visualizado na Listagem \ref{lst:dspace.webapp}, tendo seu arquivos fontes localizados no caminho da Listagem \ref{lst:dspace.fonte}. Não esqueça de conferir se o Tomcat (explicado em detalhes na Seção \ref{tomcat}) foi inicializado.

\begin{lstlisting}[language=bash, label=lst:dspace.webapp, caption=Onde o Dspace é executado.]
    /usr/local/dspace/webapps/jspui
\end{lstlisting}

\begin{lstlisting}[language=bash, label=lst:dspace.fonte, caption=Onde os arquivos fonte do Dspace estão.]
    /usr/local/dspace/webapps/jspui
\end{lstlisting}


Para acessar a página do Dspace, acesse a URL disponível na Listagem \ref{lst:dspace.url}.

\begin{lstlisting}[language=bash, label=lst:dspace.url, caption=Acessando o Dspace.]
    <ENDERECO INET>:8081/jspui
\end{lstlisting}

Se tudo rodar como esperado, a primeira tela que será visível será a Imagem \ref{img:dspace}.

\begin{figure}[H]
    \centering
    \includegraphics[width=0.5\textwidth]{../images/dspace2.png}
    \caption{Página inicial do DSpace.}
    \label{fig:dspace}
\end{figure}

\subsection{Tematres}\label{tematres}

\begin{wrapfigure}{r}{0.2\textwidth} %this figure will be at the right
    \centering
    \includegraphics[width=0.2\textwidth]{../images/tematres.jpg}
\end{wrapfigure}

Tematres é uma ferramenta para controlar vocábulos controlados e taxonomias. Suporta análise e categorização de termos de busca através de metadados, fornecendo uma interface web para realizar este controle.

Como é um sistema em PHP, ele atualmente está instalado no diretório \textit{localhost}, onde seus arquivos fontes podem ser acessados através do caminho exibido na Listagem \ref{lst:tematres.fonte}.
\begin{lstlisting}[language=bash, label=lst:tematres.fonte, caption=Arquivos fontes do tematres.]
    /var/www/localhost/htdocs/tematres
\end{lstlisting}

Para acessar a página do tematres, acesse a URL disponível na Listagem \ref{lst:tematres.url}
\begin{lstlisting}[language=bash, label=lst:tematres.url, caption=Acessando o Tematres.]
    <ENDERECO INET>/tematres
\end{lstlisting}

Se tudo rodar como esperado, a primeira tela que será visível será a Imagem \ref{img:tematres}.

\begin{figure}[H]
    \centering
    \includegraphics[width=0.5\textwidth]{../images/tematres2.png}
    \caption{Página inicial do Tematres.}
    \label{fig:dspace}
\end{figure}
